
\section{Introduction} \label{sec:introduction}
 
Compression is much older than digital computers itself\cite{wolfram2002}. With the rise of digital images in the late 1980s, however, many new algorithms and standards for image compression popped up. Today, everyone knows ZIP, MP3 and JPG, compression standards for general purpose, music and images respectively. Because of the steadily rising amount of data we produce - i.e. every minute there are 300 hours of video uploaded to the YouTube platform\footnote{\url{https://www.youtube.com/yt/press/statistics.html}, accessed 16.06.2015} - compression is still a big topic in research.

In this paper, we focus on neural networks as a means to image compression. The next section introduces neural networks and we take a look at similar work. Section \ref{sec:algorithm} describes our model and algorithm in detail. In the following section, we compare our method against two baseline algorithms, one using principle component analysis (PCA) and the second using Gaussian mixture models (GMM). Last but not least we are going to draw conclusions and interpret the results of our compression algorithm.

