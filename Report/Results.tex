
\section{Results}

\begin{figure}
\centering
\begin{tabular}{cc}
\subfloat {\includegraphics[width=4cm]{original_22}} & 
\subfloat {\includegraphics[width=4cm]{reconstructed_22}} & 
\end{tabular}
\caption{Visual comparison of original and reconstructed image.}
\label{fig:compare_images}
\end{figure}


We tested the compression algorithm with 100 images which were not in the training set and get a mean error of (......) and a compression rate of (......). 
\newline
Figure~\ref{fig:compare_images} shows an example image before any compression on the left and the reconstructed image after compression on the right. 
\newline
Idea: Test with training directly on image, and compare results with pre-training.

\subsection{Comparison to Baselines}
Describe baselines here.... \newline
Compare compression rate.... \newline
Compare error....
